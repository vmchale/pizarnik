\documentclass{article}

\usepackage{hyperref}
\usepackage{amsmath}

\begin{document}

\title{Background}
\author {V. E. McHale}
\maketitle

\tableofcontents

% $(G \oplus H)^\bot=G^\bot \& H^\bot$.

\section{Background}

Blume, Acar, and Chae present a functional language in which sum type variants (and dually, pattern match arms) are handled like row types \cite{blume2006}.

% Point being: there's lots more room for FP to win!
% https://brianmckenna.org/blog/row_polymorphism_isnt_subtyping

\section{Expression Problem}

Blume, Acar, and Chae note that they solve the expression problem for functional languages.

% also emphasizes duality

% { 0⁻¹ `true & _ `false }
% type rewriting/expansion rules... Int + a = a for a universal, inv
% insight: wildcard is drop
% inverse of a type: swap left/right on stack

% negation of types (constructive)
% inspired by linear logic?

% "right permeable" girard uses

% convince myself that "always unify everything" and putting stack variables everywhere works (greedy but also we put them everywhere soooo probably yeah)
% unlike Diggins, we also want monomorphization, hence mgu, match approach

\section{Error Hierarchies}

% https://brianmckenna.org/blog/row_polymorphism_isnt_subtyping

\bibliographystyle{plain}
\bibliography{bib.bib}

\appendix

\section{The Expression Problem}

% http://lambda-the-ultimate.org/node/4394#comment-68007

% basically same as Haskell but without the globality of typeclasses
% more "natural"; don't have to change approach entirely/start out w/ right approach

% http://lambda-the-ultimate.org/node/4394#comment-68060
% https://okmij.org/ftp/tagless-final/course/course.html

\section{Error Hierarchies}

Nick Partridge gives an example of using prisms for error hierarchies in Haskell \cite{errorprisms}.

This is dual to the lens kludge

\begin{verbatim}
class HasInt a where
    int :: Lens a Int
\end{verbatim}

which is known to be superseded by row types.

% https://blog.jle.im/entry/lenses-products-prisms-sums.html might be relevant

Both entail the globality of typeclasses and the esoteric
% https://www.tumblr.com/ezyang/62157468762/haskell-haskell-and-ghc-too-big-to-fail-panel

The Haskell solutions are more convoluted, but it is precisely this maladroitness that attests to the appetite for a solution to this problem.

\end{document}
