\documentclass{article}

\usepackage{hyperref}
\usepackage{mathpartir}
\usepackage{amsmath}

\begin{document}

\title{Typing Judgments for Stack-Based Concatenative Languages}
\author {V. E. McHale}
\maketitle

\section{Judgments I}

Lowercase Greek letters are type variables; uppercase letters are stack variables. $\Gamma$ is a context. Lowercase Latin letters are terms.
% dalet

A stack variable $A$ is a sequence of type variables $\alpha_1,\alpha_2,\ldots,\alpha_n$ of undetermined length.

\newcommand{\braces}[1]{\{~ #1 ~\}}
\newcommand{\Judge}{\Gamma\vdash}
\newcommand{\with}{~\&~}
\newcommand{\tseq}[2]{#1_1#1_2\cdots#1_#2}

\begin{mathpar}
\inferrule
    {\Gamma \vdash x : A -- B~C \\ \Gamma \vdash y : C -- D}
    {\Gamma \vdash xy : A -- B~D }
    \quad(\textsc {Cat})

\inferrule
{\Gamma \vdash x : A -- B}
{\Gamma \vdash [x] : C -- C~[A -- B]}
\quad(\textsc{Quote})

\inferrule
{\\}
{\Gamma \vdash \texttt{apply} : A~[A -- B] -- B}
\quad(\textsc{Apply})

\end{mathpar}

Note that stack variables are always leftmost, hence the inconvenient (from the perspective of the programmer) ordering of {\tt apply}.

\section{Judgments II}

% { 0⁻¹ `true & _ `false }
% type rewriting/expansion rules... Int + a = a for a universal, inv
% insight: wildcard is drop
% inverse of a type: swap left/right on stack

% negation of types (constructive)
% inspired by linear logic?

\begin{mathpar}

  \inferrule
  {\Judge x : -- \alpha}
  {\Judge x^{-1} : \alpha --}
  \quad(\textsc{Inv})

  \inferrule
  {\Judge x : -- \alpha}
  {\Judge x : -- \alpha \oplus \beta}
  \quad(\textsc{Conjunction})
  % cut? modus ponens

\inferrule
{\Judge \vdash x_1 : \alpha_1 -- A \\ \Judge x_2 : \alpha_2 -- A \\ \cdots \\ \Judge x_n : \alpha_n -- A}
{\Gamma \vdash \braces{x_1 \with x_2 \with \cdots \with x_n} : \alpha_1\oplus\alpha_2\cdots\oplus\alpha_n -- A}
\quad(\textsc{Pattern-Match})

\end{mathpar}

Note that this is not algorithmic typing, one could go awry in applying \textsc{Conjunction} spuriously.

\section{Unification}

In the presence of conjunctions,
% https://web.cecs.pdx.edu/~mpj/thih/thih.pdf

\subsection{Monomorphization}
% unlike Diggins, we also want monomorphization, hence mgu, match approach

% unlike most general unifier, we wish to have the most specific disjuctive type

% "right permeable" girard uses

% http://lambda-the-ultimate.org/node/4394#comment-68007

\end{document}
