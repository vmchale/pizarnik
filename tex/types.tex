\documentclass{article}

\usepackage{hyperref}
\usepackage{mathpartir}
\usepackage{amsmath}

\begin{document}

\title{Typing Judgments for Stack-Based Concatenative Languages}
\author {V. E. McHale}
\maketitle

\section{Judgments I}

Lowercase Greek letters are type variables; uppercase letters are stack variables. $\Gamma$ is a context. Lowercase Latin letters are terms.
% dalet

A stack variable $A$ is a sequence of type variables $\alpha_1,\alpha_2,\ldots,\alpha_n$ of undetermined length.

\newcommand{\braces}[1]{\{~ #1 ~\}}
\newcommand{\Judge}{\Gamma\vdash}
\newcommand{\with}{~\&~}
\newcommand{\tseq}[2]{#1_1#1_2\cdots#1_#2}
\newcommand{\withmany}[2]{#1_1\oplus#1_2\oplus\cdots\oplus#1_#2}
\def\tag#1{\thinspace\text{\textasciigrave} #1}

\begin{mathpar}
\inferrule
    {\Gamma \vdash x : A -- B~C \\ \Gamma \vdash y : C -- D}
    {\Gamma \vdash xy : A -- B~D }
    \quad(\textsc {Cat})

\inferrule
{\Gamma \vdash x : A -- B}
{\Gamma \vdash [x] : C -- C~[A -- B]}
\quad(\textsc{Quote})

\inferrule
{\\}
{\Gamma \vdash \texttt{apply} : A~[A -- B] -- B}
\quad(\textsc{Apply})

\end{mathpar}

Note that stack variables are always leftmost, hence the inconvenient (from the perspective of the programmer) ordering of {\tt apply}.

\section{Judgments II}

% { 0⁻¹ `true & _ `false }
% type rewriting/expansion rules... Int + a = a for a universal, inv
% insight: wildcard is drop
% inverse of a type: swap left/right on stack

% negation of types (constructive)
% inspired by linear logic?

\begin{mathpar}

  \inferrule
  {\Judge x : -- \alpha}
  {\Judge x^{-1} : \alpha --}
  \quad(\textsc{Inv})

  \inferrule
  {\Judge x : -- \alpha}
  {\Judge x : -- \alpha \oplus \beta}
  \quad(\textsc{Disjunction})
  % cut? modus ponens

\inferrule
{\Judge x_1 : A_1 \tag{a}_1 -- B \\ \Judge x_2 : A_2 \tag{a}_2 -- B \\ \cdots \\ \Judge x_n : A_n \tag{a}_n -- B}
{\Judge \braces{x_1 \with x_2 \with \cdots \with x_n} : \braces{\withmany{A\tag{a}}{n}} -- B}
\quad(\textsc{Pattern-Match})
% I think this does need to include (right-) unification of Bs...

\inferrule
{\Judge  \{\beta_i\} \subset \{\alpha_i\} \\x : A -- B~\withmany{\beta}{m} \\ \Judge y : B~\withmany{\alpha}{n} -- C}
{\Judge xy : A -- C }
\quad(\textsc{Narrow})
% TODO: break this into φ relation (which allows the left to generalize with respect to the right)

% \inferrule
% {\Judge x : \bigoplus_i \alpha -- \\ y : \bigoplus_j\beta}
% {\Judge xy : -- \bigoplus_i \left(\bigoplus_j \alpha_j\right) \beta_i}
% \quad(\textsc{Fan-Out}) (fan out with respect to...)

\end{mathpar}

Note that this is not algorithmic typing, one could go awry in applying \textsc{Disjunction} spuriously.

% We justify $\textsc{(Pattern-Match)}$ by noting that if $x_1 : \alpha_1 -- A$ and $x_2 : \alpha_2 -- A$, then $x_1 : -- \alpha_1^{-1} A$ and $x_2 : -- \alpha_2^{-1} A$, $\{x_1 \with x_2\} :$

\section{Unification}

% In the presence of disjunctions, one calculates the most specific unifier

$\alpha$ unifies with $\alpha \oplus \beta$ on the right, but not on the left---the latter would correspond to an inexhaustive pattern match. Conversely, narrowing is permitted on the left but not the right.

\subsection{Type Rewrite Rules}

If $\alpha \biguplus \beta = \gamma$, $\braces{\alpha~A \oplus \beta~B} \Rightarrow \gamma~\braces{A \oplus B}$

\subsection{Monomorphization}
% unlike Diggins, we also want monomorphization, hence mgu, match approach

% unlike most general unifier, we wish to have the most specific disjuctive type

% "right permeable" girard uses

\end{document}
