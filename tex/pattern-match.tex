%! TeX program = lualatex

\documentclass{article}

\usepackage{hyperref}
\usepackage{fontspec}
\usepackage{amsmath}

\begin{document}

\title{Pattern Matching in Pizarnik}
\author {V. E. McHale}
\maketitle

\tableofcontents

\setmonofont{DejaVu Sans Mono}[Scale=MatchAveragecase]

% $(G \oplus H)^\bot=G^\bot \& H^\bot$.

\section{Introduction}

Pizarnik's approach to pattern-matching is original, based on commitment to the idea of pattern-matching as inverse\cite{ehrenberg2009} and suggestive notation from linear logic ($\&$, ``with'', is dual to disjunction, $\oplus$).

\section{Structural Sum Types}

Tags such as \verb|`true| are Pizarnik atoms, inspired by symbols (\verb|`blue|) in k (preceded by symbols \verb|'red| in Lisp). \verb|`true| has type \verb|-- `true|. It can also be said to have type $-- \mathtt{`true}~\oplus~\mathtt{`false}$. To see why we would wish to admit $\mathtt{`true}~\oplus~\mathtt{`false}$, consider

% even then, we should only generalize on the left.
% negative polarity <-> "left permeable"
% insisting on specificity on the right=pattern-match exhaustiveness checking

\begin{verbatim}
type Bool = `true ⊕ `false;

not : Bool -- Bool
    := [ { `false⁻¹ `true & `true⁻¹ `false } ]
\end{verbatim}

\verb|`false⁻¹| has type \verb|`false --| and hence \verb|`false⁻¹ `true| has type \verb|`false -- `true|; \verb|`true⁻¹ `false| has type \verb|`true -- `false|.
% inverse type exchanges sides (sequent calculus/linear logic) `true⁻¹ : `true --
% true inverse

Perhaps counterintuitively, we get pattern-match exhaustiveness checking for free. $\oplus$ is a disjunction and thus we may generalize as we see fit, but $\&$ is a conjunction; it restricts.

To invert a sum type, we supply a product of its inverses (pattern match clauses). This is precisely the de Morgan laws.

% one can match a term of type `true against a type signature of type `true+`false, though one cannot match a term of type 1 against a type signature a
% Idea: types like row types {σ : σ -> Nil} "hasVariant" like row types idk

% $\&$ is a product; it would be farcical for a programmer to supply \verb|1| when a term of type \verb|(Int * Int)| is required.
% pattern matches are a product too...
% maybe Dhall does this with merge? lexical scoping..
% TypeScript: 1 has type 1 | String so pattern-match exhaustiveness checking is bunk.

\subsection{Reusing Pattern-Match Arms}

With tags, we could define a function

\begin{verbatim}
if : a a `true -- a
   := [ `true⁻¹ drop ]
\end{verbatim}

This is not so useful on its own---it requires the programmer to supply exactly one value and then discards it---but see how it unifies:

\begin{verbatim}
else : a a `false -- a
     := [ `false⁻¹ nip ]

choice : a a Bool -- a
       := [ { if & else } ]
\end{verbatim}

Pattern-match arms are defined just as functions and can be reused. % referenced
% one can even "cut" two in a row...
% desugar to a narrower ast (convenient)

% true inverses! `true⁻¹ is inverse to `true and has type `true --
% we get the de Morgan laws

\subsection{Wildcards}

\verb|drop : a --| is one of the building blocks of stack programming. We denote it as \verb|_| as it functions like a wildcard in pattern-matching.

\begin{verbatim}
isZero : Int -- Bool
       := [ 0⁻¹ `true & _ `false ]
\end{verbatim}

\subsection{Typesafe Head}

\begin{verbatim}
type List a = `nil ⊕ List(a) a `cons;

type NE a = List(a) a `cons;

head : NE a -- a
     := [ `cons⁻¹ nip ]
\end{verbatim}

Any nonempty list can be supplied as an argument to a function on lists. This is far more fluent than the situation in Haskell, in which \verb|foldr| \&c. for \verb|NonEmpty| must be defined separately.
% row types give us ->1 etc., "sigma types" give us

\subsection{Or-Patterns}

Suppose we have

\begin{verbatim}
type Ord = `lt ⊕ `eq ⊕ `gt;
\end{verbatim}

Then we can define

\begin{verbatim}
lte : `lt ⊕ `eq --
    := [ `lt⁻¹ & `eq⁻¹ ]
\end{verbatim}

So we get or-patterns for free and they are reusable as functions.

\subsection{Nominal Typing}

One can define

\begin{verbatim}
type Maybe a = a `just ⊕ `nothing;
\end{verbatim}

\section{Expression Problem}

The expression problem

\begin{verbatim}
type Expr = `int Int ⊕ `add Expr Expr;

eval : Expr -- Int
     := [ { `int⁻¹ & `add⁻¹ [eval] dip eval + } ]
\end{verbatim}

% matthias blume already did this: http://lambda-the-ultimate.org/node/4394

% basically same as Haskell but without the globality of typeclasses
% more "natural"; don't have to change approach entirely/start out w/ right approach

% also emphasizes duality

% { 0⁻¹ `true & _ `false }
% type rewriting/expansion rules... Int + a = a for a universal, inv
% insight: wildcard is drop
% inverse of a type: swap left/right on stack

% negation of types (constructive)
% inspired by linear logic?

% "right permeable" girard uses

% stack variables must only at leftmost (check programmer-supplied function definitions)
% unlike Diggins, we also want monomorphization, hence mgu, match approach
% http://lambda-the-ultimate.org/node/4394#comment-68007

\bibliographystyle{plain}
\bibliography{bib.bib}

\end{document}
